\input sample/plain

\newcount\xx@cnta
\newcount\xx@cntb
\newcount\xx@year
\newcount\xx@month

%%<*> \julian{年/月/日}
% 指定の日付に対する修正ユリウス通日の値をマクロ
% \thejulian に返す.
\def\julian#1{%
  \edef\xx@tmpa{#1}%
  \expandafter\xx@julian@a\xx@tmpa///\relax
}
\def\xx@julian@a#1/#2/#3/#4\relax{%
  \xx@cnta=#1\relax
  \xx@cntb=#2\relax
  \ifnum\xx@cnta<1 \xx@cnta=\@ne \fi
  \ifnum\xx@cntb<1 \xx@cntb=\@ne \fi 
  \ifnum\xx@cntb<3\relax
    \advance\xx@cntb by 12\relax
    \advance\xx@cnta by -1\relax
  \fi
  \xx@year=\xx@cnta
  \xx@month=\xx@cntb
  \multiply\xx@cnta by 1461\relax
  \divide\xx@cnta by 4\relax
  \xx@cntb=\xx@year
  \divide\xx@cntb by 100\relax  
  \advance\xx@cnta by -\xx@cntb
  \divide\xx@cntb by 4\relax
  \advance\xx@cnta by \xx@cntb
  \xx@cntb=\xx@month
  \advance\xx@cntb by -2\relax
  \multiply\xx@cntb by 520\relax
  \divide\xx@cntb by 17\relax
  \advance\xx@cnta by \xx@cntb
  \advance\xx@cnta by #3\relax
  \advance\xx@cnta by -678912
  \edef\thejulian{\number\xx@cnta}%
}

\julian{2012/12/25}
\thejulian


