\mathchardef\sum   = "103a3
\mathchardef\intop = "1222b \def\int{\intop\nolimits}
\def\sqrt{\radical"2221a}
\mathchardef\lambda= "003bb
\mathchardef\mu    = "003bc
\mathchardef\pi    = "003c0
\mathchardef\sigma = "003c3
\mathchardef\infty = "0221e
\mathchardef\plusminus = "300b1
\mathchardef\cdotp = "000b7
\def\cdots{%
  \mathinner{\cdotp\cdotp\cdotp}}


Roots of the quadratic equation $ax^2+bx+c=0$ are $x={-b\plusminus\sqrt{b^2 - 4ac} \over 2a}$.

The exponential function $e^x = \sum_{n=0}^{\infty} {x^n \over n!} = 1 + {x^1\over 1!} + {x^2\over 2!}+ \cdots$ .


日本語も通ります。

かなり複雑な確率密度関数も、%
$\int_{-\infty}^\infty {1 \over \sqrt{2\pi\sigma^2}}\hbox{exp}(- {{(x-\mu)}^2\over 2\sigma^2}) dx$
のように表示できます。

連分数はこんな感じ。$a_0+{1\over a_1+{1\over a_2+{1\over a_3+{1\over a_4}}}}$


縦長のカッコはCSS2.1では表示できないので、$_nC_{k/2} = {n\atopwithdelims() {k\over 2}}$のようにカッコそのものが大きくなります。

行列も、カッコの制限から、現状はこの程度の表示に甘んじています。

\def\matrix#1{(\halign{$##$&&$##$\cr#1})}
$$\matrix{x-\lambda & 1 & 0 \cr 
                    0 & x-\lambda & 1 \cr
                    0 & 0 & x-\lambda \cr}$$

